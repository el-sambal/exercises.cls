\question
Prove that for all $n\in\N_0$, for all finite sets $A_1,A_2,\dots,A_n$, \[\left|\bigcup_{i=1}^n A_i\right| = \sum_{\emptyset\neq J\subseteq\{1,\dots,n\}}(-1)^{|J|+1}\left|\bigcap_{j\in J}A_j\right|.\]

\begin{solution}
	Observe that the statement is true for $n=0$ (as $|\emptyset|=0$) and $n=1$ (as $|A_1|=|A_1|$ for all finite sets $A_1$).

	Now we prove the statement for $n=2$. The statement expands to \[\text{for all finite sets $A_1, A_2$:}~~~|A_1\cup A_2|=|A_1|+|A_2|-|A_1\cap A_2|.\]
	Take arbitrary finite sets $A_1, A_2$. Observe that $A_1\cup A_2=A_1\cup(A_2\setminus A_1)$. As $A_1$ and $A_2\setminus A_1$ are disjoint and $A_1$ and $A_2$ are finite, we have $|A_1\cup A_2| = |A_1|+|A_2\setminus A_1|$.

	Also observe that $A_2=(A_2\setminus A_1)\cup(A_2\cap A_1)$ and that $(A_2\setminus A_1)$ and $(A_2\cap A_1)$ are disjoint, so we have $|A_2|=|A_2\setminus A_1|+|A_2\cap A_1|$.

	Combining the two results, we find that $|A_1\cup A_2|=|A_1|+|A_2|-|A_1\cap A_2|$, which proves the case $n=2$.

	\vspace{2mm}
	We proceed using mathematical induction to prove the statement for all $n\geq2$. We have already proved the case $n=2$ and will use it as a base case.

	Assume as induction hypothesis that the statement holds for some $n=k\geq2$. We have to show that it holds for $n=k+1$. That is, we have to show that for all finite sets $A_1,A_2,\dots,A_{k+1}$, \[\left|\bigcup_{i=1}^{k+1} A_i\right| = \sum_{\emptyset\neq J\subseteq\{1,\dots,k+1\}}(-1)^{|J|+1}\left|\bigcap_{j\in J}A_j\right|.\]
	We can reach the right-hand side from the left-hand side as follows:
	\begingroup
	\allowdisplaybreaks % to allow the align* to break up over pages
	\begin{align*}
		    & \left|\bigcup_{i=1}^{k+1} A_i\right|                                                                                                                                                                                       \\
		    & \text{\red{(expanding big union)}}                                                                                                                                                                                         \\
		=~~ & \left|\left(\bigcup_{i=1}^{k} A_i\right)\cup A_{k+1}\right|                                                                                                                                                                \\
		    & \text{\red{(using base case $n=2$)}}                                                                                                                                                                                       \\
		=~~ & \left|\bigcup_{i=1}^{k} A_i\right| + |A_{k+1}| - \left|\left(\bigcup_{i=1}^{k} A_i\right)\cap A_{k+1}\right|                                                                                                               \\
		    & \text{\red{(applying distributivity of $\cap$ over $\cup$)}}                                                                                                                                                               \\
		=~~ & \left|\bigcup_{i=1}^{k} A_i\right| + |A_{k+1}| - \left|\bigcup_{i=1}^{k} (A_i\cap A_{k+1})\right|                                                                                                                          \\
		    & \text{\red{(applying induction hypothesis twice)}}                                                                                                                                                                         \\
		=~~ & \left(\sum_{\emptyset\neq J\subseteq\{1,\dots,k\}}(-1)^{|J|+1}\left|\bigcap_{j\in J}A_j\right|\right) + |A_{k+1}| - \sum_{\emptyset\neq J\subseteq\{1,\dots,k\}}(-1)^{|J|+1}\left|\bigcap_{j\in J}(A_j\cap A_{k+1})\right| \\
		    & \text{\red{(rewriting sum on the right by including $k+1$ in $J$; note the sign flip)}}                                                                                                                                    \\
		=~~ & \left(\sum_{\emptyset\neq J\subseteq\{1,\dots,k\}}(-1)^{|J|+1}\left|\bigcap_{j\in J}A_j\right|\right) + |A_{k+1}| + \sum_{\substack{J\subseteq\{1,\dots,k+1\}                                                              \\\text{s.t. }k+1\in J\\\text{and }|J|>1}}(-1)^{|J|+1}\left|\bigcap_{j\in J}A_j\right| \\
		    & \text{\red{(absorbing $|A_{k+1}|$ into sum on the right, and rewriting bounds of sum on the left)}}                                                                                                                        \\
		=~~ & \left(\sum_{\substack{\emptyset\neq J\subseteq\{1,\dots,k+1\}                                                                                                                                                              \\\text{s.t. }k+1\notin J}}(-1)^{|J|+1}\left|\bigcap_{j\in J}A_j\right|\right) + \sum_{\substack{\emptyset\neq J\subseteq\{1,\dots,k+1\}                                                                       \\\text{s.t. }k+1\in J}}(-1)^{|J|+1}\left|\bigcap_{j\in J}A_j\right| \\
		    & \text{\red{(taking both sums together)}}                                                                                                                                                                                   \\
		=~~ & \sum_{\emptyset\neq J\subseteq\{1,\dots,k+1\}                                                                       }(-1)^{|J|+1}\left|\bigcap_{j\in J}A_j\right|                                                          \\
	\end{align*}
	\endgroup

	Thus, the statement holds for $n=k+1$. Hence, the statement is proven for all $n\geq2$ by mathematical induction. We showed separately that it holds for $n\in\{0,1\}$ too. This completes the proof.
\end{solution}

\question
Let $\Sigma=\{a,b,c\}$. Prove or disprove the following statement:

\begin{center}
	\begin{minipage}{0.65\textwidth}
		\emph{``For all languages $L\in\Sigma^*$, if there exists a nondeterministic Turing machine $M$ that decides $L$ in polynomial time, then $L$ is in $\textbf{P}$.''}
	\end{minipage}
\end{center}

\begin{solution}
	The solution is left as an exercise to the reader. Send your solution to prof. P.B.M.T.
	\\\QAuthor{question author: prof. Proofstein von Beweisenlust über Mühsal bis Trauern}
\end{solution}

\question
This question asks you to find a particular number.

\begin{parts}
	\part What is the answer to the ultimate question?
	\part What is the answer to the ultimate question?
	\part What is the answer to the ultimate question?
	\part What is the answer to the ultimate question?
	\begin{subparts}
		\subpart What is the answer to the ultimate question?
		\begin{subsubparts}
			\subsubpart What is the answer to the ultimate question?
			\subsubpart What is the answer to the ultimate question?
			\subsubpart What is the answer to the ultimate question?
		\end{subsubparts}
		\subpart What is the answer to the ultimate question?
		\subpart What is the answer to the ultimate question?
	\end{subparts}
	\part What is the answer to the ultimate question?
\end{parts}

Indicate your answer here:

\begin{checkboxes}
	\choice It is 41.
	\choice It is 42.
	\CorrectChoice It is 43.
	\choice It is 44.
\end{checkboxes}

Or here:

\begin{oneparcheckboxes}
	\choice It is 41.
	\choice It is 42.
	\CorrectChoice It is 43.
	\choice It is 44.
\end{oneparcheckboxes}

\begin{solution}
	Note that, thanks to the \texttt{exam} class, the correct answer is marked with a checkmark if and only if the document is compiled with printing of the answers enabled (e.g. by passing the option \texttt{answers} to the \texttt{exercises} class).
	\QAuthor{question author: Mr. Hitchhiker}
\end{solution}

\question
This is a very long question. This is a very long question. This is a very long question. This is a very long question. This is a very long question. This is a very long question. This is a very long question. This is a very long question. This is a very long question. This is a very long question. This is a very long question. This is a very long question. This is a very long question. This is a very long question. This is a very long question. This is a very long question. This is a very long question. This is a very long question. This is a very long question. This is a very long question. This is a very long question. This is a very long question. This is a very long question. This is a very long question. This is a very long question. This is a very long question. This is a very long question. This is a very long question. This is a very long question.

This is a very long question. This is a very long question. This is a very long question. This is a very long question. This is a very long question. This is a very long question. This is a very long question. This is a very long question. This is a very long question. This is a very long question. This is a very long question. This is a very long question. This is a very long question. This is a very long question. This is a very long question. This is a very long question. 

This is a very long question. This is a very long question. This is a very long question. This is a very long question. This is a very long question. This is a very long question. This is a very long question. This is a very long question. This is a very long question. This is a very long question. This is a very long question. This is a very long question. This is a very long question. This is a very long question. This is a very long question. This is a very long question. This is a very long question. This is a very long question. This is a very long question. This is a very long question. This is a very long question. This is a very long question. This is a very long question. This is a very long question. This is a very long question. This is a very long question. This is a very long question. This is a very long question. This is a very long question.

\begin{solution}
	This solution is spread out over two pages!
	This solution is spread out over two pages!
	This solution is spread out over two pages!
	This solution is spread out over two pages!
	This solution is spread out over two pages!
	This solution is spread out over two pages!
	This solution is spread out over two pages!
	This solution is spread out over two pages!
	This solution is spread out over two pages!
	This solution is spread out over two pages!
	This solution is spread out over two pages!
	This solution is spread out over two pages!
	This solution is spread out over two pages!
	This solution is spread out over two pages!
	This solution is spread out over two pages!
\end{solution}
