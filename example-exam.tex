\documentclass[answers]{exercises}
\usepackage{amsmath}
\usepackage{amssymb}
\usepackage{hyperref}
\usepackage{comment}

% custom colors
\renewcommand\colback{red!7}
\renewcommand\coltitle{red!50!black}
\renewcommand\colbacktitle{red!5}
\renewcommand\colborder{red}
\renewcommand\ptscolor{\color{red}}
\renewcommand\QAuthorcolor{orange!90}

% custom box titles
\renewcommand\instructionsboxtitle{Instructions (READ CAREFULLY)}
\renewcommand\solutionboxtitle{Solution sketch}

% header
\newcommand{\headerleft}{Example Course (2024-2025)}
\newcommand{\headermiddle}{}
\newcommand{\headerright}{Midterm exam, page \thepage{} of \numpages}

\renewcommand\O{\mathcal O}

\begin{document}
\begin{center}
	{\LARGE Example course: Midterm Exam}

	\vspace{3mm}
	{\large Duration: two (2) hours}
\end{center}
\vspace{5mm}

\begin{instructions}
	\begin{itemize}
		\item \textbf{Write your student number at the top-right of each page! We advise to do this at the beginning of the exam. Papers without student-number will be invalid.}
		\item Write with a black or dark blue pen. \textbf{Do NOT use pencil}. Write clearly and legibly.
		      %\item You \textbf{ARE} allowed to use a simple, {non-graphical, non-programmable} calculator.
		\item The exam is {closed-book}: you are \textbf{NOT} allowed to consult any course material or notes during the exam.
		      You are \textbf{NOT} allowed to communicate with other students.
		      %\item Place your student card on the top-left corner of your table for inspection.
		\item You can obtain 100 regular points and 10 bonus points. Your grade is equal to the number of points you obtained divided by 10, with a maximum of 10.
		\item Good luck!
	\end{itemize}
\end{instructions}


\begin{questions}
	\question[42] Given a sequence of integers $A=a_1,a_2,\dots,a_n$ in which all elements are between $1$ and $m$ inclusive, suggest an $\O(n(m+\log n))$ time algorithm that computes the number of subsequences $B$ of $A$ such that $\sum B\geq\frac23\sum A$ and $\sum C<\frac23\sum A$ for all subsequences $C$ obtained by removing one element from $B$.

	\begin{solution}
		One can use dynamic programming.
		\QAuthor{Inspired by problem D from NWERC 2024.}
	\end{solution}

	\question[42] Given is a sequence of integers $A=a_1,a_2,\dots,a_n$. Suggest an algorithm that computes in $\O(n\log n)$ time the sequence $B=b_1,b_2,b_n$ where $b_i$ is the number of indices $j$ such that $j>i$ and $a_j>a_i$.

	\begin{solution}
		One can adapt the algorithm for merge sort.
	\end{solution}

	\question[16] Come up with some other interesting question to use in this example document.
	\begin{parts}
		\part[15] Maybe about segment trees?
		\part[1] Or maybe about a probabilistic algorithm?
	\end{parts}

	\begin{solution}
		Uh, I don't know?
	\end{solution}

	\bonusquestion[10] This is a bonus question!
	\begin{solution}
		Okay, nice!
	\end{solution}

\end{questions}
\end{document}
